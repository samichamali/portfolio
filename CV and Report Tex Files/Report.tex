
\documentclass[12pt, letterpaper]{article}
\usepackage{graphicx}
\usepackage[margin=0.5in]{geometry}
\usepackage{hyperref}

\begin{document}
    \begin{center}
    \textbf{Sami Chamali LaTeX CV \& Website Report}


    \end{center}
\section{Getting Access - URL's.}
    \begin{itemize}
        \item \textbf{\href{https://github.com/samichamali}{CLICK HERE}} to access Sami Chamali's Github Account
        \item \textbf{\href{https://github.com/samichamali/portfolio}{CLICK HERE}} to access the Github repository that contains the Online Portfolio's \underline{Source Code}.
        \item \textbf{\href{https://samichamali.github.io/portfolio/Images/CV.pdf}{CLICK HERE}} to access Sami Chamali's CV.
    \end{itemize}

\section{The Structure of the Portfolio and its Github Repository}
    A theme of colors black, grey, and white were used as palettes for the important parts of the website, such as the Waves, Background, and Aurora packages. I, as a developer, prefer such colors because they provide the reader a sense of \("\)Let's get to the point\("\).
    The Source code of the website, despite being somewhat large, has been well organised, having packages packed in specific directories, Images in their allocated space, and well-written javascript files. Its all present on the Github Repo Linked above.
    \par \textbf{How is the Website Structured?}
    \begin{itemize}
        \item Starts with a cool and wavy background with my name sliding down and unblurring.
        \item Provides a small introduction about me with an Image Trail package that places images behind the user's cursor as it moves.
        \item Shows an infinite planet-shaped list showing the companies I have worked with including a small description of what they are/do.
        \item Provides a peak into the LaTeX generated CV in case the reader is interested in my Professional/Educational Background.
    \end{itemize}
\section{Technologies and Packages Used}
    The React JavaScript Library was utilised along with a collection of packages collected from reactbits.dev, a popular high-quality website that offers its packages and libraries completely free of charge and free to use for any use case.
    \begin{center}
        \textbf{A total of 11 packages have been used from reactbits.dev}
    \end{center}
    \begin{enumerate}
        \item Aurora
        \item Blur Text
        \item Circular Text
        \item Gradient Text
        \item Image Trail
        \item Infinite Menu
        \item Profile Card
        \item Scroll Velocity
        \item Shiny Text
        \item Split Text
        \item Waves
    \end{enumerate}
    NOTE: These packages were utilised to power the reactive and eye-catching graphics present on the website portfolio.

\section{Reason behind the choice of the CV Structure}
    The typical general CV Structure has been used to create my CV, because it's much easier and understandable for a recruiter to skim through and pinpoint information.
    It follows a List format where each point of information goes after the other, and some points of information contain lists inside them with indentation.
    This style of indentation is much more comfortable on the eye, and provides a much clearer idea of which information belongs to which category.
    Simple and straight forward language was used while creating the CV, which helps create less clutter in cases of large lists.

\section{How is the CV Structured?}
    \begin{itemize}
        \item Starts with an Introduction explaining who I am, and provides small background information.
        \item Lists all my Past and Current Educational Achievements, containing names of Institutions, and time periods.
        \item Shows my Previous and Current Work Experience, naming the job position, explaining its role, naming the company, and locating it.
        \item Languages Spoken, informs the reader of all the languages I speak and my level in all of them based on certified language tests.
        \item Displays my Previous Projects, Naming the Project and what it does.
    \end{itemize}
\section{Reflection on Weaknesses and Strengths}
    Strenghts
    \begin{itemize}
        \item The CV is extremely clear to read, providing the recruiter with straight forward information and exactly what they are looking for.
        \item The Website is modern and minimalistic, which is fits in with modern design standards and themes.
        \item The Use of packages and frameworks, can prove to recruiters that I'm capable of easily integrating and manipulating packages to function or look a certain way.
    \end{itemize}
    Weaknesses
    \begin{itemize}
        \item The CV is still weak in terms of professional background, which severely hinders my chances of competing with others in today's job market.
        \item The Website is extremely easy to navigate by a regular modern user, but might be a bit hard to navigate for a not very experienced or old user.
        \item The Website needs requires a better contact me section, rather than just pasting contact information in the CV itself.
    \end{itemize}

\end{document}
